% ---
% RESUMOS
% ---

% RESUMO em português
\setlength{\absparsep}{18pt} % ajusta o espaçamento dos parágrafos do resumo
\begin{resumo}

O presente projeto tem como objetivo principal o desenvolvimento de uma plataforma virtual \textit{open-source} voltada para a simulação da dinâmica de um \textit{CubeSat,} com especial ênfase nos elementos críticos do subsistema de determinação e controle de atitude, proporcionando uma interface gráfica intuitiva. Para alcançar essa meta, pretende-se elaborar roteiros detalhados partindo da instalação e configuração das aplicações computacionais, seguindo para a modelagem físico-matemática do \textit{CubeSat}, sensores e atuadores seguindo \textit{Unified Robotic Description Format (URDF)}, a aplicação de técnicas de determinação e controle de atitude com auxilio de \textit{Python, C++ }e \textit{GNU Octave,} assim como a criação de um ambiente espacial virtual \textit{Simulation Description Format (SDF)}, interligados pelo \textit{Robot Operating System (ROS)}, e \textit{Gazebo Simulator.}

Considerando a importância do estudo experimental na engenharia, a plataforma \textit{CubeSat} foi concebida com a meta inicial de facilitar a interação dos estudantes universitários com o espaço e a tecnologia. No entanto, o desenvolvimento e lançamento de um \textit{CubeSat} são tarefas desafiadoras, envolvendo recursos consideráveis. O custo associado, notadamente na casa dos milhões de reais para uma única missão, pode representar uma parcela significativa do orçamento institucional, exemplificado pelos 1\% do orçamento total da Universidade Federal do ABC (UFABC) em 2020. Nesse contexto, o projeto propõe uma solução intermediária.

Guiado pela missão de despertar a vocação científica, inserir estudantes em atividades de pesquisa e aprimorar o desempenho dos alunos na engenharia, o projeto alinha-se às transformações educacionais aceleradas pela pandemia de COVID-19. Utilizando aplicações \textit{open-source} para modelagem, programação e simulação, a proposta não apenas visa desenvolver uma ferramenta técnica avançada, mas principalmente cultiva uma abordagem educacional que promove a participação ativa dos estudantes na exploração do espaço e na engenharia aeroespacial.

Espera-se que o projeto, ao atingir os objetivos propostos, resulte em uma plataforma funcional para desenvolvimento colaborativo, contribuindo significativamente para a formação prática dos estudantes de engenharia aeroespacial, ao facilitar a exploração do espaço de maneira acessível e inovadora.


 \textbf{Palavras-chaves}: CubeSat. Simulação. Determinação e Controle de Atitude. Plataforma virtual. Código Aberto.
\end{resumo}

% ABSTRACT in english
\begin{resumo}[Abstract]
 \begin{otherlanguage*}{english}
 
This project aims to develop an open-source virtual platform dedicated to simulating the dynamics of a CubeSat, with a particular focus on critical elements of the attitude determination and control subsystem, providing an intuitive graphical interface for this purpose. To achieve this goal, detailed scripts will be devised, encompassing the installation and configuration of computational applications, as well as the physical-mathematical modeling of the CubeSat, sensors, and actuators following the Unified Robotic Description Format (URDF). Additionally, it involves applying attitude determination and control techniques using Python, C++, and GNU Octave, and creating a virtual spatial environment with the Gazebo Simulator in the Simulation Description Format (SDF), interconnected through the Robot Operating System (ROS).

Recognizing the significance of experimental study in engineering, the CubeSat platform was initially designed to facilitate university students' interaction with space and technology. However, the development and launch of a CubeSat pose challenging tasks requiring substantial resources. The associated cost, potentially reaching millions of Brazilian reais for a single mission, can represent a significant portion of institutional budgets, as exemplified by 1\% of the total budget of the Federal University of ABC (UFABC) in 2020. In this context, the project proposes an intermediary solution.

Guided by the mission to awaken scientific vocation, engage students in research activities, and enhance their performance in engineering, the project aligns with educational transformations accelerated by the COVID-19 pandemic. Utilizing open-source applications for modeling, programming, and simulation, the proposal not only aims to develop an advanced technical tool but primarily nurtures an educational approach that fosters active student participation in space exploration and aerospace engineering.

The project anticipates yielding a functional platform for collaborative development, significantly contributing to the practical education of aerospace engineering students by facilitating accessible and innovative space exploration.

   \vspace{\onelineskip}
 
   \noindent 
   \textbf{Keywords}:     cubeSat. simulation. attitude determination and control. virtual platform. opensource.
 \end{otherlanguage*}
\end{resumo}