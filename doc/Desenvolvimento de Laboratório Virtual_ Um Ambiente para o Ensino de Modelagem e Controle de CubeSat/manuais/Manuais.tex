
\section*{Figuras}\label{sec:figuras}
\addcontentsline{toc}{section}{figuras}

Este documento segue as normas estabelecidas pela~\citeonline[3.1-3.2]{NBR6028:2003}.

As normas da~\citeonline[3.1-3.2]{NBR6028:2003} especificam que o caption da figura deve vir abaixo da mesma.

A Figura~\ref{fig:log} ilustra...

\begin{figure}[htpb]
   \centering
   \includegraphics[scale=.3]{figs/logo}
   \caption{Breve explicação sobre a figura. Deve vir abaixo da mesma.}
   \label{fig:log}
\end{figure}

\section*{Tabelas}\label{sec:tabelas}
\addcontentsline{toc}{section}{tabelas}

A Tabela~\ref{tab:tabela} apresenta os resultados...

\begin{table}[htpb]
   \centering
   \caption{Breve explicação sobre a tabela. Deve vir acima da mesma.}\label{tab:tabela}
   \begin{tabular}{|l|c|c|c|c|c|c|r|}
        \hline
        \small{XX} & \small{FF} & \small{PP} & \small{YY} & \small{Yr} & \small{xY} & \small{Yx} & \small{ZZ} \\ \hline
               615 &    18      &     2558   &    0,9930  &    0,9930  &    0,9930  &    0,9930  &    0,9930  \\ \hline
               615 &    18      &     2558   &    0,9930  &    0,9930  &    0,9930  &    0,9930  &    0,9930  \\ \hline
               615 &    18      &     2558   &    0,9930  &    0,9930  &    0,9930  &    0,9930  &    0,9930  \\ \hline
               615 &    18      &     2558   &    0,9930  &    0,9930  &    0,9930  &    0,9930  &    0,9930  \\ \hline
               615 &    18      &     2558   &    0,9930  &    0,9930  &    0,9930  &    0,9930  &    0,9930  \\ \hline
   \end{tabular}
\end{table}

\chapter{Uso de referências bibliográficas}
% ---

A formatação das referências bibliográficas conforme as regras da ABNT são um
dos principais objetivos do \abnTeX. Consulte os manuais
\citeonline{abntex2cite} e \citeonline{abntex2cite-alf} para obter informações
sobre como utilizar as referências bibliográficas.

%-
\subsection{Acentuação de referências bibliográficas}
%-

Normalmente não há problemas em usar caracteres acentuados em arquivos
bibliográficos (\texttt{*.bib}). Na~\autoref{tabela-acentos} você encontra alguns exemplos das conversões mais importantes. Preste atenção especial para `ç' e `í'
que devem estar envoltos em chaves. A regra geral é sempre usar a acentuação
neste modo quando houver conversão para letras maiúsculas.

\begin{table}[htbp]
\caption{Tabela de conversão de acentuação.}
\label{tabela-acentos}

\begin{center}
\begin{tabular}{ll}\hline\hline
acento & \textsf{bibtex}\\
à á ã & \verb+\`a+ \verb+\'a+ \verb+\~a+\\
í & \verb+{\'\i}+\\
ç & \verb+{\c c}+\\
\hline\hline
\end{tabular}
\end{center}
\end{table}


% ---
\section{Deu pau em algo?}
% ---

Consulte a FAQ com perguntas frequentes e comuns no portal do \abnTeX:
\url{https://code.google.com/p/abntex2/wiki/FAQ}.

Inscreva-se no grupo de usuários \LaTeX:
\url{http://groups.google.com/group/latex-br}, tire suas dúvidas e ajude a galera se tiver tudo certo.



