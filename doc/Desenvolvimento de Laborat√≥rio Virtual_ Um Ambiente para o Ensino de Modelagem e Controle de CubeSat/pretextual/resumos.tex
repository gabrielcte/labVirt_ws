% ---
% RESUMOS
% ---

% RESUMO em português
\setlength{\absparsep}{18pt} % ajusta o espaçamento dos parágrafos do resumo
\begin{resumo}

O presente projeto tem como objetivo principal o desenvolvimento de uma plataforma virtual \textit{open-source} voltada para a simulação da dinâmica de um \textit{CubeSat,} com especial ênfase nos elementos críticos do subsistema de determinação e controle de atitude, proporcionando uma interface gráfica intuitiva. Para alcançar essa meta, pretende-se elaborar roteiros detalhados partindo da instalação e configuração das aplicações computacionais, seguindo para a modelagem físico-matemática do \textit{CubeSat}, sensores e atuadores em \textit{Extensible Markup Language (XML)}, a aplicação de técnicas de determinação e controle de atitude com auxilio de \textit{Python, C++ }, assim como a representação de um ambiente espacial virtual interligados pelo \textit{JSBSim}, e \textit{FlightGear}.

Considerando a importância do estudo experimental na engenharia, a plataforma \textit{CubeSat} foi concebida com a meta inicial de facilitar a interação dos estudantes universitários com o espaço e a tecnologia. No entanto, o desenvolvimento e lançamento de um \textit{CubeSat} são tarefas desafiadoras, envolvendo recursos consideráveis. O custo associado, notadamente na casa dos milhões de reais para uma única missão, pode representar uma parcela significativa do orçamento institucional, exemplificado pelos 1\% do orçamento total da Universidade Federal do ABC (UFABC) em 2020. Nesse contexto, o projeto propõe uma solução intermediária.

Guiado pela missão de despertar a vocação científica, inserir estudantes em atividades de pesquisa e aprimorar o desempenho dos alunos na engenharia, o projeto alinha-se às transformações educacionais aceleradas pela pandemia de COVID-19. Utilizando aplicações \textit{open-source} para modelagem, programação e simulação, a proposta não apenas visa desenvolver uma ferramenta técnica avançada, mas principalmente cultiva uma abordagem educacional que promove a participação ativa dos estudantes na exploração do espaço e na engenharia aeroespacial.

Espera-se que o projeto, ao atingir os objetivos propostos, resulte em uma plataforma funcional para desenvolvimento colaborativo, contribuindo significativamente para a formação prática dos estudantes de engenharia aeroespacial, ao facilitar a exploração do espaço de maneira acessível e inovadora.


 \textbf{Palavras-chaves}: CubeSat. Simulação. Determinação e Controle de Atitude. Plataforma virtual. Código Aberto.
\end{resumo}

% ABSTRACT in english
\begin{resumo}[Abstract]
 \begin{otherlanguage*}{english}
 	
 The main objective of this project is the development of an open-source virtual platform aimed at simulating the dynamics of a CubeSat, with a special emphasis on the critical elements of the attitude determination and control subsystem, providing an intuitive graphical interface. To achieve this goal, we intend to elaborate detailed guides starting from the installation and configuration of computational applications, followed by the physical-mathematical modeling of the CubeSat, sensors, and actuators in Extensible Markup Language (XML), the application of attitude determination and control techniques with the help of Python and C++, as well as the representation of a virtual space environment interconnected by JSBSim and FlightGear.
 
 Considering the importance of experimental study in engineering, the CubeSat platform was conceived with the initial goal of facilitating the interaction of university students with space and technology. However, developing and launching a CubeSat is a challenging task involving considerable resources. The associated costs, notably in the millions of reais for a single mission, can represent a significant portion of the institutional budget, exemplified by the 1% of the total budget of the Federal University of ABC (UFABC) in 2020. In this context, the project proposes an intermediate solution.
 
 Guided by the mission to awaken scientific vocation, involve students in research activities, and improve student performance in engineering, the project aligns with the educational transformations accelerated by the COVID-19 pandemic. Using open-source applications for modeling, programming, and simulation, the proposal not only aims to develop an advanced technical tool but also cultivates an educational approach that promotes active student participation in space exploration and aerospace engineering.
 
 It is expected that the project, upon achieving its proposed objectives, will result in a functional platform for collaborative development, significantly contributing to the practical training of aerospace engineering students by facilitating space exploration in an accessible and innovative manner.

   \vspace{\onelineskip}
 
   \noindent 
   \textbf{Keywords}:     cubeSat. simulation. attitude determination and control. virtual platform. opensource.
 \end{otherlanguage*}
\end{resumo}