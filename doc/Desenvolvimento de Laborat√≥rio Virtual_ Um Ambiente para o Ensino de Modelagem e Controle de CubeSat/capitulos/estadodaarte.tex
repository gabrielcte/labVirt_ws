\chapter{Estado da Arte}\label{cap:estArte}

Este capítulo fornece uma revisão dos trabalhos relacionados e resume a contribuição da presente dissertação de mestrado para o estado da arte.

A contribuição primária desta dissertação, como delineada na introdução, é o desenvolvimento de uma plataforma open-source para facilitar a aprendizagem de modelagem, simulação e controle de CubeSats.

Atualmente, os laboratórios virtuais para o ensino de engenharia, como robótica, modelagem, simulação e controle, alcançaram um estágio avançado, principalmente devido à adoção de ferramentas open-source como ROS (Robotic Operating System), Gazebo e Octave. O ROS oferece uma infraestrutura flexível e modular para o desenvolvimento de sistemas robóticos, permitindo a integração de diferentes componentes e algoritmos de controle. Por sua vez, o Gazebo é um simulador de robótica 3D que proporciona a criação de ambientes virtuais realistas para testes e experimentação. Complementarmente, o Octave oferece uma poderosa plataforma de computação numérica para análise e implementação de algoritmos de controle.

Essas ferramentas combinadas possibilitam aos estudantes e pesquisadores explorarem de maneira prática e eficaz os conceitos de modelagem, simulação e controle de CubeSats, contribuindo assim para o avanço da educação e pesquisa na área aeroespacial.

No entanto, é importante ressaltar que as ferramentas do ROS e Gazebo não são tão amplamente utilizadas pela comunidade aeroespacial devido à sua limitação em simular nativamente condições como microgravidade e ambientes sem fricção, elementos essenciais para a análise de CubeSats no espaço.

A seguir, serão discutidos alguns trabalhos relacionados ao uso de ferramentas open-source como ROS, Gazebo ou Octave para o ensino de modelagem ou controle de robótica ou CubeSats.

\section*{Trabalhos Relacionados}\label{sec:primTrab}
\addcontentsline{toc}{section}{Trabalhos Relacionados}

\citeonline{lotfi2021use} abordaram um estudo de caso que analisa os potenciais e as limitações das aplicações de código aberto, destacando a modelagem, simulação e análise de um Motor de Corrente Contínua. De acordo com o artigo, cerca de 90\% dos estudantes recorrem ao GNU Octave ou ao Simulador Gazebo, com essa porcentagem reduzida para 70\% entre os educadores, reafirmando a preferência da comunidade por essas duas ferramentas abertas. A conclusão dessa primeira parte ressalta que o GNU Octave oferece vantagens, como funções e bibliotecas úteis para controle, modelagem e análise numérica.

\citeonline{lotfi2022use}  sintetizaram, em outro estudo de caso, os potenciais e as limitações das aplicações de código aberto na implementação do controle de um manipulador robótico com 2 graus de liberdade. Novamente, o GNU Octave é mencionado por apresentar resultados satisfatórios, mas a falta de um ambiente gráfico para simulações é destacada como uma limitação. Nesse estudo, essa lacuna é superada com o uso do Simulador Gazebo em conjunto com o ROS (Robotic Operating System), embora seja observado que essa ferramenta pode ser menos amigável para iniciantes.

\citeonline{linner2011modeling} descreveram o uso do ROS e Gazebo para a modelagem e simulação de robôs, destacando a utilidade da simulação cinética e dinâmica na avaliação da interação humana com construções robóticas. Embora o artigo se refira principalmente a construções terrestres, o conceito de construção robótica pode ser aplicado também a estruturas como a Estação Espacial Internacional.

Diante do aumento do interesse em manipuladores em órbita, \citeonline{hao2021intelligent} identificaram uma lacuna nas técnicas atuais de modelagem e controle de manipuladores robóticos em ambiente espacial. Sugeriram, portanto, uma arquitetura e plataforma para a realização de estudos nessa área, apresentando uma arquitetura inteligente de GNC (Guiamento, Navegação e Controle) de espaçonave com componentes de IA de ponta para manipulação em órbita. Para treinar a plataforma, tornou-se necessário o acesso a dados confiáveis, os quais foram identificados como indisponíveis ou de custo elevado. Como solução, desenvolveram um simulador visual utilizando a ferramenta Unreal Engine 4, denominado OrVIS (Orbital Visualization). Para testar o sistema, utilizaram o ROS em conjunto com o Gazebo, ressaltando que, como um teste de bancada, não integrava visualização e dinâmica espacial.

\citeonline{kang2019urdf} também reconheceu o aumento na demanda por manipuladores e, consequentemente, por simulações envolvendo esses dispositivos. Considerando que o Gazebo e o ROS são plataformas consolidadas no mercado, ele propôs automatizar a geração de URDFs (Unified Robotic Description Format), o formato utilizado por essas aplicações, a partir dos graus de liberdade dos manipuladores. O URDF é um XML (extensible Markup Language) utilizado para descrever modelos robóticos em ambientes de simulação, especialmente no contexto do ROS. Esse tipo de arquivo contém informações detalhadas sobre a geometria, cinemática, dinâmica e outras propriedades físicas de um robô, incluindo links, juntas, sensores e suas relações espaciais. Esses arquivos desempenham um papel fundamental na modelagem e simulação de robôs em ambientes virtuais, possibilitando que os desenvolvedores visualizem, controlem e testem seus sistemas robóticos antes de implementá-los no mundo real.

\citeonline{udugama2023mini}, é apresentado um projeto de localização e mapeamento simultâneo de um robô móvel utilizando o simulador Gazebo em conjunto com o ROS. É evidente que uma das grandes vantagens desse framework é a diversidade de bibliotecas, plugins e implementações disponíveis para sensores de distância e visualização, odometria, motores e modelagem de robôs.

\citeonline{canas2020ros} é uma plataforma de acesso aberto projetada para a aprendizagem prática de robótica inteligente em cursos de engenharia. Conhecida como Robotics-Academy, ela compreende uma coleção de 18 exercícios apresentando diversos robôs, como carros e drones. A plataforma utiliza o middleware Robotic Operating System (ROS) e o simulador 3D Gazebo, juntamente com a linguagem de programação Python. Dada a natureza multidisciplinar da robótica, a Robotics Academy enfatiza mais os algoritmos do que o middleware.

Como mencionado anteriormente e aprofundado em \citeonline{canas2020open}, destaca-se o curso prático à distância de drones, com ênfase na programação desses Veículos Aéreos Não Tripulados (VANTs). O hardware, que representa o próprio robô, é completamente simulado pelo Gazebo, enquanto os softwares dos drivers para a leitura das informações dos sensores e o controle dos atuadores são fornecidos pelo ROS. Os estudantes desenvolvem algoritmos a partir de templates disponibilizados. Os exercícios abordam uma variedade de tarefas, como navegação por posição, seguir objetos e pouso em um carro em movimento, entre outros.

Um aspecto que merece destaque na Robotics Academy, conforme abordado por \citeonline{roldan2022ros}, é que em uma de suas atualizações mais recentes, foram oferecidas ainda mais opções para distribuição e utilização da plataforma. Inicialmente e até hoje, é possível obter a plataforma baixando seu código-fonte pelo GitHub. No entanto, percebeu-se que a instalação e configuração de pacotes extras eram uma barreira. Para superar esse obstáculo, foram apresentadas duas soluções: a instalação via Docker, que já inclui todas as configurações prévias e pode ser utilizada em sistemas Linux, Windows e MacOS; e também, para eliminar qualquer complicação de configuração por parte do usuário, foi disponibilizada a opção de um webserver pronto para uso, onde a computação ocorre de forma transparente para o usuário. Essas soluções visam maximizar a distribuição entre os interessados.

\citeonline{ramon2023task} propõem várias tarefas de controle espacial para objetos em órbita complexos e com vários graus de liberdade. Apresentam uma framework unificada para simulações de robôs espaciais chamada OnOrbitROS, baseada no ROS. O desenvolvimento e teste de um sistema robótico em seus estágios iniciais são essenciais para reduzir custos, e as simulações estão se tornando cada vez mais comuns para isso. Assim a plataforma OnOrbitROS contribui para a comunidade aeroespacial, que tem sido relutante em adotar o ROS/Gazebo, ao oferecer uma solução que incorpora as qualidades de implementação e teste rápidos que essas ferramentas oferecem.

Esses são apenas alguns exemplos de trabalhos que exploram o uso de ferramentas open-source para o ensino de modelagem e controle de CubeSats. Existem muitos outros estudos e projetos interessantes nessa área, demonstrando o crescente interesse e importância dessas tecnologias na educação e pesquisa espacial.
