% ----------------------------------------------------------
% Introdução 
% Capítulo sem numeração, mas presente no Sumário
% ----------------------------------------------------------

\chapter*[Introdução]{Introdução}
\addcontentsline{toc}{chapter}{Introdução}

A plataforma CubeSat emergiu como uma alternativa valiosa para países em desenvolvimento ampliarem seu acesso ao espaço. No caso do Brasil, desde 2014, a Agência Espacial Brasileira (AEB) tem colaborado com institutos de pesquisa e universidades na utilização de CubeSats, \cite{Santos2018}. Diferentemente do contexto dos Estados Unidos, onde os CubeSats são frequentemente empregados como projetos estudantis, no Brasil, o desenvolvimento desses satélites muitas vezes se torna uma parte significativa do programa espacial do país.

No entanto, o desenvolvimento e lançamento de um CubeSat não são tarefas trivialmente acessíveis, demandando recursos consideráveis. O custo associado de R\$ 2.921.129,50 para uma única missão \cite{CGEE2018}, pode representar uma parcela significativa do orçamento de uma instituição de ensino, como exemplificado pelos 1\% do orçamento total da Universidade Federal do ABC (UFABC) em 2020 \cite{wikipedia2023}.

Essa realidade destaca a dificuldade em atingir os objetivos iniciais dos CubeSats, que incluíam proporcionar aos estudantes universitários a oportunidade de interagir com o espaço e a tecnologia relacionada. Diante desse cenário, surge a necessidade premente de encontrar uma solução intermediária que permita o treinamento, ensino e experimentação de forma eficaz para estudantes de engenharia aeroespacial e áreas afins.

Uma alternativa promissora para preencher essa lacuna prática é a utilização de aplicações de modelagem, programação e simulação. Contudo, a escolha entre softwares proprietários e open-source torna-se crucial, considerando os custos e a praticidade. Para contornar essas questões, exemplos de sucesso, como a Robotic Academy no ensino de robótica, demonstram a viabilidade de plataformas web de educação gratuita e aberta que facilitam o aprendizado prático \cite{canas2020ros}.

Nesse contexto, o presente projeto de pesquisa propõe a criação de um laboratório virtual utilizando ferramentas open-source. Este laboratório abrangerá modelagem, programação, controle e simulação de sensores, ambiente espacial e atuadores, focando especificamente na plataforma CubeSat.

Além disso, o cenário global atual, marcado pela pandemia de COVID-19, acelerou a adoção de métodos de ensino remoto. Estratégias como o home office e o ensino a distância redefiniram as fronteiras entre o mundo digital e o real, proporcionando uma nova perspectiva para a educação. Este projeto se alinha a essas transformações, oferecendo uma solução educacional híbrida e acessível, que integra a visualização do comportamento mecânico de modelos de CubeSats, experimentação em ambientes virtuais e reais, direcionada a estudantes de engenharia aeroespacial, controle, mecânica, professores e entusiastas \cite{bolu2020engineering}.


\section*{Motivação}\label{sec:Motivação}
\addcontentsline{toc}{section}{Motivação}

Em diversas áreas da engenharia, o estudo experimental é fundamental para o desenvolvimento de habilidades práticas essenciais para a resolução eficiente de problemas futuros. Os laboratórios híbridos surgem como uma abordagem acessível, possibilitando uma interação iterativa entre o mundo real e virtual, local e a distância. 

Guiado pela missão de despertar a vocação científica e incentivar novos talentos entre estudantes de graduação, bem como contribuir significativamente para a formação e inserção desses estudantes em atividades de pesquisa, desenvolvimento tecnológico e inovação.

Esta proposta visa incorporar práticas e tecnologias de ensino híbrido para aprimorar o desempenho dos alunos na engenharia, proporcionando um contato prático abrangente que os laboratórios tradicionais muitas vezes não conseguem oferecer.

Não somente ao se desenvolver uma ferramenta técnica avançada, mas principalmente ao se cultivar uma abordagem educacional que promova a participação ativa dos estudantes na exploração do espaço e na engenharia aeroespacial, alinhando-se assim aos princípios fundamentais da formação acadêmica e científica.


\section*{Objetivos}\label{sec:objetivos}
\addcontentsline{toc}{section}{Objetivos}


\subsection*{Objetivo Geral}\label{sec:Objetivo Geral}
\addcontentsline{toc}{section}{Objetivo Geral}

\begin{itemize}
    \item Criar uma plataforma virtual open-source com capacidade de simulação da dinâmica de um CubeSat, enfocando os elementos cruciais do subsistema de determinação e controle de atitude, e proporcionando uma interface gráfica intuitiva.
\end{itemize}

\subsection*{Objetivos Específicos}\label{sec:Objetivos Específicos}
\addcontentsline{toc}{section}{Objetivos Específicos}
\begin{itemize}
    \item Redigir um Script de Instalação e Configuração das Aplicações Computacionais.
    \item Criar modelo físico-matemático e 3D de um CubeSat.
    \item Criar modelo de Sensores, Atuadores.
    \item Criar algorítimo de determinação e controle de atitude.
    \item Criar ambiente espacial.
    \item  Criar Simulação do CubeSat em ambiente virtual.
    \item Redigir roteiros de modelagem e simulação como experimentos.
\end{itemize}
