\chapter{Implementando Laboratório Virtual}\label{cap:proposta}

O presente projeto propõe o desenvolvimento de um laboratório virtual como de ambiente para o ensino de modelagem e controle de CubeSats por meio de simulação. Utilizando ferramentas aberta e livres contornando o problema de custo. E de roteiros para instalação, configuração e uso, contornando o problema da curva de aprendizado.

\section{Requisitos}\label{Requisitos}

Para a realização do presente projeto está foi escolhido as aplicações a seguir:

\begin{itemize}
    \item Ubuntu 24.04 LTS
    \item JSBSim
    \item FlightGear
    \item Python
    \item Blender
\end{itemize}

O JSBSim é uma plataforma de código aberto para modelagem de dinâmica de voo, simulando a física e a matemática dos 6 graus de liberdade do movimento de aeronaves, foguetes e outros veículos aéreos. Ela pode ser executada independentemente ou integrada a outros simuladores como o FlightGear, \citeonline{manualJSBSim}.

O FlightGear por sua vez é um simulador de voo gratuito e de código aberto que oferece uma experiência realista de voo para várias plataformas incluindo o Windows, macOS e Linux podendo ser usado para pesquisa. Quando integrado ao JSBSim ele oferece uma interface visual, permitindo visualizar o comportamento de uma aeronave simuladas, \citeonline{manualFlightGear}.

Ambos programas são escritos em C++, mas possuem integração com o Python, os modelos das aeronaves são descritos em XML, onde é incluído propriedades inerciais, modelos aerodinâmicos, sistema e componentes.

Para o funcionamento é previsto o hardware mínimo a seguir:

\begin{itemize}
    \item Memória: 4 GB
    \item Disco: SSD 120 GB
    \item Placa de Vídeo: 1GB Dedicada
    \item Processador: QuadCore 2.1 GHz
\end{itemize}

É possível encontrar em suas respectivas páginas oficiais de forma, o sistema operacional Ubuntu, o modelo de dinâmica de voo JSBSim e o simulador de voo FlightGear.

\section{Considerações Finais}



